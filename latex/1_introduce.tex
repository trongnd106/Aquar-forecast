\section{Giới thiệu}
\subsection{Lí do chọn đề tài}
Vùng biển ven bờ Việt Nam là khu vực có điều kiện tự nhiên thuận lợi cho nuôi trồng và khai thác nhiều loài thủy sản có giá trị kinh tế cao, trong đó tiêu biểu là cá giò và hàu. Các loài này không chỉ đóng vai trò quan trọng trong phát triển kinh tế biển và sinh kế của người dân ven biển, mà còn góp phần thúc đẩy ngành nuôi trồng thủy sản theo hướng hàng hóa và xuất khẩu. Tuy nhiên, hiệu quả và tính bền vững của hoạt động nuôi trồng thủy sản phụ thuộc chặt chẽ vào chất lượng môi trường nước, vốn chịu tác động đồng thời của nhiều yếu tố vật lý, hóa học và sinh học.

Các thông số môi trường như oxy hòa tan (DO), pH, độ mặn, nhiệt độ, nồng độ các chất dinh dưỡng và chất ô nhiễm (NH₃, H₂S, TSS, BOD₅, kim loại nặng, v.v.) có ảnh hưởng trực tiếp đến quá trình sinh trưởng, sức khỏe, khả năng sinh sản và tỷ lệ sống của cá giò và hàu. Trong bối cảnh biến đổi khí hậu, nước biển dâng, thay đổi chế độ mưa – dòng chảy, cùng với sự gia tăng của các hoạt động xả thải công nghiệp, nông nghiệp và đô thị ven biển, các điều kiện môi trường biển ngày càng biến động phức tạp và khó dự đoán. Các phương pháp đánh giá truyền thống dựa trên quan trắc đơn lẻ hoặc phân tích thống kê đơn giản thường không đủ khả năng phản ánh đầy đủ tính động, tính chu kỳ và mối quan hệ phi tuyến giữa các yếu tố môi trường theo không gian và thời gian. Do đó, việc ứng dụng các phương pháp học máy và phân tích dữ liệu hiện đại vào bài toán đánh giá và dự báo chất lượng môi trường nước là cần thiết và có ý nghĩa thực tiễn cao. Thông qua việc khai thác dữ liệu quan trắc lịch sử, các mô hình học máy có thể hỗ trợ dự báo sớm rủi ro môi trường, xác định vùng nuôi có mức độ phù hợp cao, đồng thời cung cấp cơ sở khoa học cho việc ra quyết định trong quản lý, quy hoạch và phát triển nuôi trồng thủy sản theo hướng bền vững.

\subsection{Nội dung thực hiện}
Trong khuôn khổ dự án, nhóm tập trung xây dựng một quy trình phân tích và dự báo chất lượng môi trường nước nhằm phục vụ đánh giá mức độ phù hợp sinh cảnh (Habitat Suitability Index – HSI) cho nuôi trồng thủy sản, cụ thể là đối với hai đối tượng nghiên cứu chính: hàu và cá giò. Quy trình được thiết kế theo hướng tổng hợp, kết hợp giữa phân tích dữ liệu môi trường, mô hình học máy và đánh giá sinh thái.

Trước hết, dữ liệu quan trắc môi trường nước theo quý tại các trạm ven biển được thu thập và tiến hành tiền xử lý. Các bước tiền xử lý bao gồm chuẩn hóa định dạng thời gian, làm sạch dữ liệu, xử lý các giá trị bị thiếu thông qua nội suy và thống kê đặc trưng, loại bỏ hoặc làm giảm ảnh hưởng của các giá trị ngoại lai, cũng như xây dựng các đặc trưng độ trễ (lag features). Việc sử dụng các đặc trưng độ trễ cho phép mô hình phản ánh được tính chất chuỗi thời gian, xu hướng và chu kỳ mùa vụ của các yếu tố môi trường tại từng trạm quan trắc.

Trên cơ sở dữ liệu đã được xử lý, nhóm xây dựng và huấn luyện các mô hình học máy dựa trên thuật toán XGBoost để dự báo các biến môi trường theo quý. Các mô hình được chia thành hai nhóm chính: nhóm mô hình dự báo các kim loại nặng (như CN, As, Cd, Pb, Cu, Hg, Zn, Total Cr) và nhóm mô hình dự báo các biến môi trường không phải kim loại (như DO, pH, nhiệt độ, độ mặn, NH₃, BOD₅, TSS, v.v.). Phương pháp dự báo cuốn chiếu (rolling forecast) được áp dụng nhằm dự báo nhiều quý trong tương lai cho từng trạm, trong đó kết quả dự báo của các quý trước được sử dụng làm đầu vào cho các bước dự báo tiếp theo.

Kết quả dự báo các yếu tố môi trường được sử dụng để tính toán chỉ số HSI dựa trên các ngưỡng sinh học đặc trưng cho từng loài. Chỉ số HSI được xây dựng trên cơ sở chuẩn hóa từng biến môi trường về thang [0,1] theo mức độ phù hợp sinh học, sau đó tổng hợp để đánh giá mức độ phù hợp tổng thể của môi trường nuôi theo không gian và thời gian. Thông qua chỉ số này, nhóm có thể xác định các giai đoạn và khu vực có điều kiện môi trường thuận lợi, cận thuận lợi hoặc không phù hợp cho nuôi trồng hàu và cá giò. Bên cạnh đó, nhóm còn thiết kế và xây dựng một giao diện trực quan dưới dạng bản đồ vùng nuôi trồng, cho phép hiển thị kết quả dự báo và chỉ số HSI theo từng trạm và từng quý trong tương lai. Giao diện này giúp người dùng dễ dàng theo dõi, so sánh và đánh giá mức độ phù hợp môi trường giữa các khu vực, từ đó hỗ trợ ra quyết định trong công tác quản lý và quy hoạch vùng nuôi.

Toàn bộ dự án được triển khai bằng ngôn ngữ Python, với giao diện người dùng được xây dựng bằng thư viện Streamlit. Mã nguồn được thiết kế theo hướng mô-đun, cho phép dễ dàng cập nhật mô hình với dữ liệu mới, mở rộng cho nhiều trạm quan trắc hoặc nhiều đối tượng nuôi khác nhau, đồng thời hỗ trợ phân tích và trực quan hóa dữ liệu phục vụ nghiên cứu khoa học và quản lý nuôi trồng thủy sản.


