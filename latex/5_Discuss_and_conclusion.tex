\section{Kết luận}
\subsection{Kết quả thu được}

Trong khuôn khổ dự án, nhóm đã xây dựng và triển khai thành công một quy trình phân tích và dự báo chất lượng môi trường nước ven biển nhằm phục vụ đánh giá mức độ phù hợp sinh cảnh (Habitat Suitability Index -- HSI) cho nuôi trồng thủy sản, với hai đối tượng nghiên cứu chính là hàu và cá giò. Quy trình được thiết kế theo hướng tổng hợp, kết hợp giữa xử lý dữ liệu chuỗi thời gian, mô hình học máy và đánh giá sinh thái, từ đó hình thành một hệ thống phân tích tương đối hoàn chỉnh.

Về mặt dữ liệu, nhóm đã thu thập và xử lý dữ liệu quan trắc môi trường nước theo quý tại các trạm ven biển. Các bước tiền xử lý được thực hiện một cách hệ thống, bao gồm chuẩn hóa định dạng thời gian, làm sạch dữ liệu, xử lý các giá trị dưới ngưỡng phát hiện, nội suy và xử lý dữ liệu thiếu, giảm ảnh hưởng của các giá trị ngoại lai, cũng như xây dựng các đặc trưng độ trễ. Việc sử dụng các đặc trưng độ trễ (lag features) cho phép mô hình nắm bắt được xu hướng, tính chu kỳ và sự phụ thuộc theo thời gian của các yếu tố môi trường tại từng trạm quan trắc.

Trên cơ sở dữ liệu đã được xử lý, nhóm đã xây dựng và huấn luyện các mô hình học máy dựa trên thuật toán XGBoost để dự báo các biến môi trường theo quý. Các mô hình được chia thành hai nhóm chính: nhóm mô hình dự báo các kim loại nặng (CN, As, Cd, Pb, Cu, Hg, Zn, Total Cr) và nhóm mô hình dự báo các biến môi trường không phải kim loại (DO, pH, nhiệt độ, độ mặn, NH$_3$, BOD$_5$, TSS, v.v.). Kết quả đánh giá trên tập huấn luyện cho thấy các mô hình đạt sai số dự báo ở mức chấp nhận được, với giá trị RMSE trung bình của các biến môi trường chính dao động quanh ngưỡng từ khoảng 1.4 đến 1.6, tùy theo từng biến và từng đối tượng nuôi. Điều này cho thấy mô hình có khả năng học được các mối quan hệ phi tuyến giữa các yếu tố môi trường và phản ánh tương đối tốt xu hướng biến động theo thời gian.

Phương pháp dự báo cuốn chiếu (rolling forecast) được áp dụng nhằm dự báo nhiều quý liên tiếp trong tương lai cho từng trạm quan trắc. Thông qua phương pháp này, kết quả dự báo của các quý trước được sử dụng làm đầu vào cho các bước dự báo tiếp theo, cho phép mô hình mô phỏng kịch bản biến động môi trường theo thời gian dài hơn và phản ánh tính liên tục của chuỗi thời gian.

Dựa trên kết quả dự báo các yếu tố môi trường, nhóm tiến hành tính toán chỉ số HSI cho từng đối tượng nuôi trồng. Mỗi biến môi trường được chuẩn hóa về thang giá trị từ 0 đến 1 dựa trên các ngưỡng sinh học đặc trưng cho hàu và cá giò, sau đó được tổng hợp để xác định mức độ phù hợp tổng thể của môi trường nuôi. Kết quả cho thấy chỉ số HSI có sự phân hóa rõ rệt theo không gian và thời gian, với các giá trị HSI trung bình dao động trong khoảng từ mức cận thuận lợi đến thuận lợi tại nhiều trạm ven biển trong một số giai đoạn, trong khi một số khu vực và thời điểm khác thể hiện mức HSI thấp, phản ánh điều kiện môi trường kém phù hợp cho nuôi trồng.

Ngoài ra, nhóm đã xây dựng một giao diện trực quan dưới dạng bản đồ vùng nuôi trồng, cho phép hiển thị kết quả dự báo các biến môi trường và chỉ số HSI theo từng trạm và từng quý trong tương lai. Giao diện này giúp người dùng dễ dàng theo dõi sự thay đổi của điều kiện môi trường, so sánh mức độ phù hợp sinh cảnh giữa các khu vực khác nhau, từ đó hỗ trợ quá trình đánh giá và ra quyết định trong quản lý nuôi trồng thủy sản.

\subsection{Đánh giá kết quả và hướng phát triển}

Tổng thể, các kết quả đạt được cho thấy quy trình phân tích và dự báo chất lượng môi trường nước được xây dựng trong dự án là khả thi và phù hợp với mục tiêu nghiên cứu. Các mô hình XGBoost thể hiện khả năng dự báo ổn định đối với các biến môi trường chính, với sai số RMSE ở mức chấp nhận được đối với dữ liệu chuỗi thời gian theo quý. Việc tích hợp kết quả dự báo vào chỉ số HSI cho phép chuyển đổi các kết quả học máy thành thông tin sinh thái có ý nghĩa thực tiễn, hỗ trợ đánh giá mức độ phù hợp môi trường cho nuôi trồng hàu và cá giò.

Tuy nhiên, dự án vẫn còn tồn tại một số hạn chế. Thứ nhất, độ chính xác của mô hình phụ thuộc lớn vào chất lượng và độ đầy đủ của dữ liệu quan trắc, trong khi một số biến môi trường chưa có dữ liệu thực tế đầy đủ và phải được bổ sung bằng dữ liệu tổng hợp. Thứ hai, phương pháp dự báo cuốn chiếu có thể dẫn đến hiện tượng tích lũy sai số khi dự báo trong nhiều bước thời gian liên tiếp, làm giảm độ tin cậy của kết quả trong dài hạn. Thứ ba, chỉ số HSI hiện tại được xây dựng dựa trên các ngưỡng sinh học cố định, chưa phản ánh đầy đủ khả năng thích nghi sinh thái theo từng khu vực và điều kiện môi trường cụ thể.

Trong các hướng phát triển tiếp theo, dự án có thể được mở rộng theo nhiều hướng. Một là, tích hợp thêm dữ liệu quan trắc mới, dữ liệu khí tượng -- hải văn và các yếu tố tác động từ con người nhằm nâng cao độ chính xác của mô hình dự báo. Hai là, nghiên cứu và so sánh thêm các mô hình học sâu cho chuỗi thời gian, chẳng hạn như LSTM hoặc Transformer, để đánh giá khả năng cải thiện sai số RMSE và hạn chế tích lũy sai số trong dự báo dài hạn. Ba là, cải tiến phương pháp xây dựng chỉ số HSI theo hướng động, cho phép điều chỉnh ngưỡng sinh học theo không gian và thời gian. Cuối cùng, hệ thống trực quan có thể được phát triển thành một công cụ hỗ trợ ra quyết định hoàn chỉnh, phục vụ hiệu quả cho công tác quản lý và quy hoạch vùng nuôi trồng thủy sản trong thực tế.


